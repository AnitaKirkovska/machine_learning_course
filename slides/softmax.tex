\documentclass[12pt]{article}

\usepackage{amsmath,amsthm,amssymb,hyperref,fullpage,bm,palatino}

\newtheorem{thm}{Theorem}
\newtheorem{cor}{Corollary}
\newtheorem{lem}{Lemma}
\newtheorem{rem}{Remark}
\newtheorem{dfn}{Definition}

\newcommand{\R}{\mathbb{R}}

\newcommand{\x}{\boldsymbol{x}}
\newcommand{\X}{\boldsymbol{X}}
\newcommand{\y}{\boldsymbol{y}}
\newcommand{\yh}{\hat{\boldsymbol{y}}}
\newcommand{\w}{\boldsymbol{w}}
\newcommand{\A}{\boldsymbol{A}}

\newcommand{\mse}{\mathrm{MSE}}

\title{Softmax and categorical cross entropy loss}

\author{Pawel Wocjan}

\begin{document}

\maketitle


\begin{abstract}
We define softmax and categorical cross entropy for multiclass classification 
\end{abstract}

\section{Classification}

\begin{tabular}{|c|c|c|}
\hline
Problem type & Last-layer activation & Loss function \\ \hline \hline
Binary classification & \texttt{sigmoid} & \texttt{binary\_crossentropy} \\ \hline
Multiclass, single-label classification & \texttt{softmax} & \texttt{categorical\_crossentropy} \\ \hline
Multiclass, multi-label classification & \texttt{sigmoid} & \texttt{binary\_crossentropy} \\ \hline
\end{tabular}

\section{Softmax activation function}

Let $z_1,\ldots,z_m$ be the weighted inputs of the $m$ neurons of the last layer. Each of the neurons corresponds to one of the $m$ classes of the multiclass classification problem at hand. We consider the single-label situation. We convert the weighted input vector $\boldsymbol{z}=(z_1,\ldots,z_m)^T\in\R^m$ into a probability vector $\boldsymbol{a}=(a_1,\ldots,a_m)^T\in\R^n$ by applying the so-called softmax activation function. 

For $k\in [m]$, the activation $a_k$ of the $k$th neuron is defined as follows:
\begin{equation}
a_k = \frac{e^{z_k}}{\sum_{j=1}^n e^{z_j}}.
\end{equation}
It is straightforward to verify that this yields a probability distribution. The values $a_k$ are all positive because the range of the exponential function is $(0,\infty)$. They sum up to $1$ because of the normalization in the denominator.

Using the product and chain rule, we can show that
\begin{equation}\label{eq:aoverz}
{\partial a_k \over \partial z_j} = a_k \cdot (\delta_{jk} - a_j),
\end{equation}
where $\delta_{jk}$ is the so called Kronecker delta, which is equal to $1$ if $j=k$ and $0$ if $j\neq k$.

\section{Cross entropy}

Let $\boldsymbol{p}=(p_1,\ldots,p_m)$ and $\boldsymbol{q}=(q_1,\ldots,q_m)$ be two probability distributions. In information theory, the cross entropy is defined by
\begin{equation}
H(\boldsymbol{p},\boldsymbol{q}) = - \sum_{k=1}^m p_k \log q_k.
\end{equation}
See \url{https://en.wikipedia.org/wiki/Cross_entropy} for a quick overview.

\section{Categorical cross entropy loss function}

Let $y\in[m]$ be a label. Using the so-called one-hot or categorical encoding, we construct a corresponding vector $\boldsymbol{y}=(y_1,\ldots,y_m)^T\in\R^n$ such that $m-1$ of its entries are equal to $0$ and exactly one entry is equal to $1$. The position of the entry $1$ is given by the label $y$. For instance, for $m=3$ classes, label $1$ corresponds to $(1,0,0)$, the label $2$ to $(0,1,0)$, and the label $3$ to $(0,0,1)$.

Assume that the feature vector $\x$ produces the activation vector $\boldsymbol{a}=(a_1,\ldots,a_m)$ in the last layer. Assume that the correct label is $y$. Then the categorical cross entropy loss $\mathcal{L}$ is defined by
\begin{equation}
\mathcal{L} = -\sum_{k=1}^m y_k \log a_k,
\end{equation}
where $\y=(y_1,\ldots,y_m)$ is the categorical encoding of $y$. Observe that $\mathcal{L}=H(\y,\boldsymbol{a})$.

The partial derivatives of $\mathcal{L}$ with respect to $a_k$ is
\begin{equation}
{\partial \mathcal{L} \over \partial a_k} = -\frac{y_k}{a_k}.
\end{equation}

The partial derivatives of $\mathcal{L}$ with respect to $z_j$ is
\begin{eqnarray}
{\partial \mathcal{L} \over \partial z_j} 
& = &
\sum_{k=1}^m {\partial \mathcal{L} \over \partial a_k} \cdot {\partial a_k \over \partial z_j} \\
& = &
\sum_{k=1}^m -\frac{y_k}{a_k} \cdot a_k \cdot (\delta_{kj} - a_j)  \\
& = &
\sum_{k=1}^m y_k \cdot (a_j - \delta_{kj})
\end{eqnarray}

\section{Simple neural network with softmax activation and categorical cross entropy}

Let us consider a network that takes feature vectors of the form $\x=(x_1,\ldots,x_n)^T\in \R^n$ as input and has $m$ output neurons with softmax activation.  Then for $j\in[m]$, the weighted input of the $j$th neuron is given by
\begin{equation}
z_j = \sum_{i=1}^n w_{ji} x_i + b_j.
\end{equation}
The weights and the bias of $j$th neuron are $(w_{j1},\ldots,w_{jn})$ and $b_j$, respectively.  The weighted inputs $z_j$ are transformed into probabilities by the softmax activation.

\medskip
\noindent
Compute the partial derivatives of the categorical cross entropy loss function the weights and biases. You will need these to implement stochastic gradient descent for the homework. This should be very easy. The more difficult part was computing the partial derivatives of the categorical entropy loss function with respect to the weighted inputs.  

\end{document}